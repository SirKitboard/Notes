\documentclass{report}
\usepackage{amsmath}

\title{\textbf{Lab 1 - Electric Field Plotting}}
\date{2016-02-03}
\author{L11\\ Nauman Shahzad\\TA: Yifan Fang}

\newcommand\tab[1][1cm]{\hspace*{#1}}

\begin{document}
\maketitle

\begin{flushleft}
{\Large \textbf{\underline{Introduction:}}}
\\\hspace{.1in}\\
{\large
 \tab This lab is about plotting ``equipotential lines", lines of equal potential difference, for two different 2D charge distributions.
After plotting the equipotential lines, we then plotted down the electric field lines which are perpendicular to the equipotential lines, then solve for the electric field values for both charge distributions and their uncertainty measurements.
The equipment utilized for the lab were as follows:
\begin{itemize}
\item 1 digital voltmeter
\item 2 voltage probes (One was fixed, we fixed down the negative one, and one with a right-angle handle that was moved around to find voltages)
\item Insulated wire leads (2 red and 2 black ones), with banana plugs
\item 2 carbonized acetate sheets (One with "parallel plate" and one wiht "electric dipole" geometry) placed on platform and used at the two different 2D charge distributions
\item 1 3V battery
\item 1 meter stick
\item 2 photocopied sheets similar looking to carbonized acetate sheets so as to record data found
\end{itemize}
}

\newpage

{\Large \textbf{\underline{Procedure:}}}
\\
{\large
\begin{enumerate}
\item Connect the battery, terminals of the sheet and voltmeters with insulated wires
\item Turn on the voltmeter, set it to 20V, and make sure to turn on battery as well
\item Place one of the carbonized acetate sheets on the platform
\item Align and place the terminal clamps on boths sides of the carbonized sheet, make sure to be on ends of either parallel plates or dipoles, whichever sheet you are using first
\item Place the stationary probe on same plate or dipole that the negative terminal is placed upon
\item Move around the positive moving probe and record voltage values and sketch out the equipotential lines
\item After recording about 7 lines do the same with the other sheet that was not used and record data for that as well
\item Calculate the Electric Field values for both charge distributions using the data gathered \[ E = \frac{\Delta V}{\Delta d}\]
\end{enumerate}
}

\newpage

{\Large \textbf{\underline{Data:}}}
\\
{\large
\begin{flushleft}

\[\sigma \Delta V = \sqrt{(\sigma V)^2 + (\sigma V)^2}\]
\[\sigma \Delta d = \sqrt{(\sigma d)^2 + (\sigma d)^2}\]
\[E = \frac{\Delta V}{\Delta d}\]
\[\frac{\sigma E}{E} = \sqrt{(\frac{\sigma \Delta V}{\Delta V})^2 - (\frac{\sigma \Delta d}{\Delta d})^2}\]
\[\sigma V = \pm .005V\]
\[\sigma \Delta V = \pm .007V\]
\[\Delta d = .01m\]
\[\sigma d = \pm .0005m\]
\[\sigma \Delta d = \pm .0007m\]
\\\hspace{.1in}\\
{\Large \textbf{\hspace{.8in}Parallel Plates \hspace{.3in} Dipoles}}
\begin{equation}
\begin{split}
E = 42J/C &\hspace{.5in} E = 32J/C
\\
\Delta V = .42V &\hspace{.5in} \Delta V = .32V
\\
\sigma E = \pm 1.98J/C  &\hspace{.5in} \sigma E = \pm 2.13J/C
\end{split}
\end{equation}
\end{flushleft}
}

\newpage

{\Large \textbf{\underline{Analysis:}}}
\\\hspace{.1in}\\
{\large
\tab The value for the Electric Fields, based on data gathered during experiment, with their uncertainty measurements are \(E = 42 \pm 1.98J/C\) and
\(E = 32 \pm 2.13J/C\) for the parallel plates and dipoles. One reason for the error in measurment being smaller for parallel plates than dipoles is that it was easier to plot the equipotential lines for the plates, mostly straight lines except for the edges, whereas all of the equipotential lines, except the center one, were curved and harder to find and plot.
\\
\tab The equipotential lines contain points all of which have the same voltage, the electric fields sketched on the photocopies depict the direction of the charge from positive to negative. The equipotential lines used to calculate \(\Delta V\) and \(\Delta d\) were the lines near the center/middle of the plot. These lines were the easiest of the seven lines to plot, which would help lead to a smaller error in measurement with the calculations.
}

\newpage

{\Large \textbf{\underline{Conclusion:}}}
\\\hspace{.1in}\\
{\large
\tab Some ways to help improve the lab would be to instead of having a probe that we have to hold up and steady to move around to find the equipotential lines it would be better to have it set up similar to the stationary probe. This way all that we need to do is just move the positive probe around the carbonized acetate sheet to find the equipotential lines, which would help one be able to sketch better plots. Another way would be to have fresh set of carbon sheets for each session of the experiment, because our carbon sheets were scratched up and could have affected some of the voltage values that we recorded.
\\
\tab This lab helped to illustrate the charge distributions between two different 2D charge distributions and the Electric Fields that result from these different distributions.
}

\end{flushleft}
\end{document}